%!TEX root = ../MainBody.tex

% 第三章
\chapter{系统实现}% 使用\cite{}命令引用数据库中文献
\section{运行平台}
\begin{itemize}
    \item MacOS操作系统
    \item 开发工具:Visual Studio Code
    \item 浏览器:Chrome、Safari
    \item 编程语言:HTML5+SCSS+JavaScript(ECMAScript 6.0)
    \item 数据库:MongoDB
\end{itemize}
\section{功能界面}
\subsection{登录界面}
\subsection{主界面(图书列表界面)}
\section{代码}
\subsection{项目结构}
\begin{lstlisting}[numberstyle=\tiny,keywordstyle=\color{blue!70},
    commentstyle=\color{red!50!green!50!blue!50},frame=shadowbox,
    rulesepcolor=\color{red!20!green!20!blue!20},basicstyle=\ttfamily]
    |- dist // webpack 打包输出目录
    |- config    // webpack 构建配置目录  
    ​    |- tpl  
    ​        |- index.html 入口  
    ​    |- webpack.config.js    //  webpack 配置文件  
    |- node-modules  //  公共依赖  
    |- node-server  //  node 端  
    ​    |- main.js  
    ​    |- middleware // 中间件  
    ​    |- proxy.config.js // 前后端proxy配置   
    ​    |- route // 路由配置  
    |-- web-client 
    ​    |- common 公共资源,如图片、SVG等  
    ​    |- components   // 公共组件  
    ​    |- domain   // 处理网络请求   
    ​    |- model    // 页面与api请求之间的业务中间层  
    ​    |- styles   // 公共样式  
    ​    |- utils  //公共工具  
    ​    |- page  //页面 
    ​        |- A   // 页面A 
    ​            |- index.tsx  // 页面逻辑  
    ​            |- store.ts  // mobX 数据管理   
    ​        |- B  
    ​            |- index.tsx  
    ​            |- store.ts  
    |- package.json   
    |- .gitignore  
    |- README.md   
    |- stylelintrc.js // eslint 代码格式检查配置文件  
    |- tslint.json // tslint 代码格式检查配置文件   
    |- theme.js  // 修改antd主题色   
    |- tsconfig.json    
\end{lstlisting}
\subsection{关键代码}
界面框架及静态路由:

\begin{lstlisting}[numberstyle=\tiny,keywordstyle=\color{blue!70},
    commentstyle=\color{red!50!green!50!blue!50},frame=shadowbox,
    rulesepcolor=\color{red!20!green!20!blue!20},basicstyle=\ttfamily]
export default class AppWrapper extends React.Component {
  render() {
      return (
      <HashRouter>
          <AppLayout>
              <Route exact path="/" component={BookList} >
              </Route>
              <Route path="/me" component={Me}></Route>
              <Route path="/bookList" component={BookList}>
              </Route>
          </AppLayout>
          
      </HashRouter>
      )
  }
    }    
\end{lstlisting}
视图展示:

\begin{lstlisting}[numberstyle=\tiny,keywordstyle=\color{blue!70},
    commentstyle=\color{red!50!green!50!blue!50},frame=shadowbox,
    rulesepcolor=\color{red!20!green!20!blue!20},basicstyle=\ttfamily]
    @observer
export default class BookList extends Component {
    onSearch = (val: string) =>{
        BookStore.queryBookList(val);
    }
    render() {
        const { keyword, bookList } = BookStore;
        return (
            <Layout className="container">
            <Content>
                <List
                dataSource={bookList}
                renderItem={
                  (item: IBook) => <BookItem key={item.id
                  }
                item={item}/>}>
                </List> 
            </Content>         
            </Layout>
        )
    }
}
\end{lstlisting}
数据处理:

\begin{lstlisting}[numberstyle=\tiny,keywordstyle=\color{blue!70},
    commentstyle=\color{red!50!green!50!blue!50},frame=shadowbox,
    rulesepcolor=\color{red!20!green!20!blue!20},basicstyle=\ttfamily]
    @action
    queryBookList = async () => {
      const pageNo = this.pagination.current || 1;
      const params = {
        ...this.keyword,
        pageNo,
        pageSize: PAGE_SIZE,
      }
      this.tableLoading = true;
      const res = await getBookList(params);
      if (res && res.data.code === 200) {
        const { data, page } = res.data;
        if (pageNo > 1) {
          this.bookList.push(...data);
        } else {
          this.bookList = data;
        }
        this.pagination.total = page.totalCount;
      }
      this.tableLoading = false;
    };
\end{lstlisting}
网络请求层:

\begin{lstlisting}[numberstyle=\tiny,keywordstyle=\color{blue!70},
    commentstyle=\color{red!50!green!50!blue!50},frame=shadowbox,
    rulesepcolor=\color{red!20!green!20!blue!20},basicstyle=\ttfamily]
export const getBookList = async(keyword: string) => (
await get('/api/bookList', {
        params: {
                keyword,
                }
    }));
\end{lstlisting}
Node层

\begin{lstlisting}[numberstyle=\tiny,keywordstyle=\color{blue!70},
    commentstyle=\color{red!50!green!50!blue!50},frame=shadowbox,
    rulesepcolor=\color{red!20!green!20!blue!20},basicstyle=\ttfamily]
{
    url: '/api/abtest/list',
        async controller(response, thriftExec) {
            const ret = await thriftExec(ABAdminService,
            'queryABConfigs',
            {
                arg0: this.query,
            });
            response.sendJSON(
              normalize(ret, {
                dataPath: 'items',
                pagePath: 'page',
              }),
            );
          },
}   
\end{lstlisting}

CSRF拦截

\begin{lstlisting}[numberstyle=\tiny,keywordstyle=\color{blue!70},
    commentstyle=\color{red!50!green!50!blue!50},frame=shadowbox,
    rulesepcolor=\color{red!20!green!20!blue!20},basicstyle=\ttfamily]
export default async function(next) {
  if (this.req.method === 'GET') {
    this.cat.logEvent('CSRF-GET', this.req.url);
    return await next;
  }
  let referer = this.req.headers['referer'] || '';
  if (!referer) {
    this.cat.logEvent('CSRF-NoRef', this.req.url);
    return await next();
  }
  if (referer) {
    this.cat.logEvent('CSRF-Ref', referer);
  }
  referer = url.parse(referer);
  if (
    whiteList.some(domain => {
      return domain.test(referer.hostname);
    })
  ) {
    return await next;
  }
  this.cat.logEvent('CSRF-Forbidden',
  referer && referer.hostname);
  this.status = 403;
  this.body = 'Forbiddened';
}
\end{lstlisting}