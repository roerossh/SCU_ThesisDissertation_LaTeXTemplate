%!TEX root = ../MainBody.tex

% 第一章
\chapter{绪论}
% 以上为一级标题——绪论,此处为一级标题正文。参考文献引自\cite{tlc}。
% \section{二级标题}
% 此处为二级标题正文。
\section{背景}
现有的图书管理/租借系统大多并不支持线上借书服务。这意味着读者在网页端或微信端预约借书后,需要到馆内才能取得书籍。这一方式存在一些问题,如寒暑假期间、读者出差在外、由于学业或工作繁忙等原因,无法亲自到馆借书,希望能够通过图书快递的方式,直接将书籍配送到读者所在地址。
这一门槛可能把大多数有阅读需求的读者拒之门外,本系统研发的初衷便在于解决这一问题,打通物流,跨越读者与图书的空间障碍,改善借阅方式,打造"读者-web端-物流-图书"模式,满足读者的多元化需求。另外,由于阅读具有非强制性,大部分租借系统也存在利用率低下的问题,部分可能从中受益的读者可能并没有接触使用,可以通过扩展系统的社交属性,将消息推送至主流社交媒体等方式,一定程度上缓解这一问题。
\section{选题意义}
\begin{enumerate}
    \item 有利于提高纸质图书利用率。在校内图书馆系统现有功能基础上,增加图书异地租借服务,支持图书邮寄,主要解决假期期间学生借书困难的问题,同时也方便有校外借书需求的学生和教职工,进一步提高图书馆海量资源的利用率。
    \item 有利于提高师生综合素养。图书是知识、信息传播的重要载体,通过阅读能不断完善个人知识体系,建立正确的人生观、世界观、价值观。本系统支持pc端和移动端web访问,采用流式布局,界面简洁友好。在基础功能实现的基础上优化用户体验,是获取图书资源的优质渠道。有一定推广阅读的作用,配合图书馆系统现有的推荐服务,发挥图书其潜在价值,对提升师生阅读综合素养有所裨益。
    \item 有利于合理开发与配置信息资源,实现图书服务多元化。图书租借系统下要求对图书信息资源的开发与配置具备新观念和新方式。有利于结合当下图书借阅的实际情况,切实提升图书馆的服务水平和服务质量;其次,有助于多层次、联合式地开发和利用信息资源。从重视理论性资源开发逐步过渡到重视实践性和生活性的信息资源开发,从而可以满足用户的不同信息需求。
    \item 有利于推动图书相关研究进展。本系统可为相关大数据挖掘及分析工作提供可靠真实的图书借阅数据,推动此类研究的推进和发展。    
\end{enumerate}
\section{国内外研究现状}
近年来,随着计算机技术和网络技术的迅速崛起,互联网、“互联网+”模式的广泛应用日渐深刻的改变着人们的生产生活方式。互联网逐步进入传统的流通领域,电子商务开始流行,而物流则是电子商务中的重要一环。如今发达的互联网物流系统大大便捷了人们的日常生活,利用它的强大特性,作为图书租借系统的支持,能有效提高资源的流通和利用。

​国内外数字图书馆的实践活动大致可以分为以下三种类型: 服务研究型、资源服务型,联合建设型。虽然,从严格意义上讲,资源服务型不能算是数字图书馆,但它的网上信息服务目前已自大多数图书馆开展,是现阶段我国图书界提供网上数字服务的主要形式。

​目前,国内浙江省杭州图书馆和广东省深圳罗湖区图书馆已开通快递借书服务,读者可同时采用线上或线下方式借还书,另外还提供了读者优先选书功能,把图书采购和选择权在一定范围内交给读者,研发智慧图书系统,加强多平台信息共享,增加与读者之间的沟通交流,同时对读者的借阅数据进行分析,更好地明确读者的需求,主动推送更切合需求的书目。市民可通过图书馆借阅卡、实名认证或评估芝麻信用等方式进行认证,认证成功即可预约借书。

