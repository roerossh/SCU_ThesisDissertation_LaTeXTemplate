%!TEX root = ../MainBody.tex

% 第三章
\chapter{结论}% 使用\cite{}命令引用数据库中文献
\section{软件测试}
\subsection{测试用例001:登录}
测试目标:用户登录流程

初始设置:未登录状态下进入系统主界面;若已登录,则退出重新登录
\begin{table}[ht]
    \centering
    \begin{tabular*}{0.9\textwidth}{p{0.4\textwidth}p{0.3\textwidth}p{0.2\textwidth}}
        \hline
        输入及操作 & 预期结果 & 实际结果 \\
        \hline
        输入不存在的账号ghostUser,密码123456,登录 & 提示账号不存在 & 通过 \\
        输入正确账号user01,错误密码123455,登录    & 提示账号或密码输入错误 & 通过 \\
        输入正确账号user01,正确密码123456,登录 &  登录成功,跳转至主界面 & 通过 \\
        点击注册按钮 &  提示到馆注册借书卡,使用借书卡账号密码登录 & 通过 \\ 
        \hline
    \end{tabular*}
\end{table}
\subsection{测试用例002:图书列表}
测试目标:图书列表展示正常

初始设置:使用读者账号成功登录系统
\begin{table}[ht]
    \centering
    \begin{tabular*}{0.9\textwidth}{p{0.3\textwidth}p{0.4\textwidth}p{0.2\textwidth}}
        \hline
        输入及操作 & 预期结果 & 实际结果 \\
        \hline
        进入主界面 & 展示加载圈,书籍加载成功加载圈消失,列表数据展示正常,每个列表项正常显示图片、馆藏数量、可借数量、借书按钮。 & 通过 \\
        假网状态下进入主界面 & 展示加载圈,5秒之后弹出"数据加载失败"提示,列表页展示空数据样式 & 通过 \\
        点击单个列表项的详情按钮 &  详情上方展示气泡,气泡内为该书籍的简介 & 通过 \\
        点击分页按钮 &  页面展示具体页的数据,样式和数量均正确 & 通过 \\ 
        点击导航栏tab,点击返回 &  页面成功跳转,点击返回按钮后返回上一页 & 通过 \\ 
        \hline
    \end{tabular*}
\end{table}
\newpage
\subsection{测试用例003:读者借书}
测试目标:借书流程正常,读者能够成功预约借书

初始设置:使用读者账号成功登录界面,且用户未处于欠费状态
\begin{table}[ht]
    \centering
    \begin{tabular*}{0.9\textwidth}{p{0.35\textwidth}p{0.35\textwidth}p{0.2\textwidth}}
        \hline
        输入及操作 & 预期结果 & 实际结果 \\
        \hline
        点击列表项的借书按钮 & 借书成功,预约列表显示该书籍 & 通过 \\
        点击进入预约列表,点击列表项的取消预约 & 取消借书成功,预约列表不显示该书籍 & 通过 \\
        点击借书按钮预约成功后,管理员已将该书籍邮寄 &  待还列表中显示该书籍,并显示寄出单号,预约列表不显示该书籍 & 通过 \\
        \hline
    \end{tabular*}
\end{table}
\newpage
\subsection{测试用例004:用户还书}
测试目标:还书流程正常,读者能够成功还书

初始设置:使用读者账号成功登录界面,待还列表中存在待还书籍,且该书籍已经邮寄
\begin{table}[h]
    \centering
    \begin{tabular*}{0.9\textwidth}{p{0.4\textwidth}p{0.3\textwidth}p{0.2\textwidth}}
        \hline
        输入及操作 & 预期结果 & 实际结果 \\
        \hline
        点击待还列表,选择待还书籍,点击"还书"按钮,输入物流单号 & 物流单号保存成功,列表页显示该单号 & 通过 \\
        点击填写过物流单号的待还书籍,点击"修改物流"按钮,输入物流单号 & 物流单号修改成功,列表页显示修改后的单号 & 通过 \\
        管理员收到图书,修改书籍状态之后,点击待还列表 &  待还列表中不再显示该书籍 & 通过 \\
        \hline
    \end{tabular*}
\end{table}
\newpage
\section{总结}
到此,毕业设计进入收尾阶段,大学四年本科也即将结束了。在完成毕业设计的过程中,收获颇丰:

熟悉软件工程的流程和规范,从需求分析、概要设计、详细设计、编码、测试到交付、验收、维护,在导师的指导下独立完成整个项目。前期进行了一系列调研工作,阅读国内外图书馆系统和物流分发系统的期刊和文献,从用户的角度进行需求分析,给出用例分析和活动图,根据用户使用步骤和交互过程初步确立基本模块和界面设计初稿;再根据需求分析得出的结果进行概要设计,将本系统划分为读者模块和管理员模块,进一步将读者模块细分为登录、浏览、租借、个人等子模块,将管理员模块细分为登录、借书处理、还书处理等子模块,实现功能上的解耦。并对系统的基本处理流程、组织结构、数据结构、接口和异常情况进行设计,为下一步的详细设计做铺垫。在详细设计时,对各个模块的数据流向、视图更新、交互触发、接口模型、层次结构和调用关系等作出详细描述。根据详细设计给出的规约,进行编码实现。而在编码实现过程中,遇到了不少前期设计未考虑到的缺陷,如用户点击借书按钮,在网络延迟、浏览器卡顿或异常暴力操作的情况下,可能一次借书需求,实际却点击多次借书按钮,后端应该对此类请求做幂等操作,以保证数据和交互的一致性、稳定性。在遇到缺陷时候,我又返回到需求分析步骤,分析排查此类情况,给出方案进行处理。

熟悉编程语言,了解行业新兴技术。在实际编码中,我用到了ECMAScript 6.0、React、Webpack、Typescript、mobX等前端新技术,不得不说部分新技术的学习曲线是十分陡峭的,且遵从"二八定律",想要学得更加深入还需要花更多的时间去研究、去实践。有两类观点,一是去掌握基础、研究原生代码;二是掌握新技术、学习新框架,显然在限定时间内二者很难兼得,两种观点任取其一都是不对的,唯有以黄金分割办的比例,两者都涉猎。知识虽说触类旁通,但是不同知识具有不同的细节规范,一不留神就会花大量时间去排查错误,区分细节。对于这一点,毫无办法,不如采用胡适先生"怕什么真理无穷,进一步有进一步的欢喜"的态度坦然面对。毕业设计是大学生涯的重点,也是终身学习,人生新篇章的起点。

了解项目搭建流程,掌握发现问题、快速处理问题的能力。使用Webpack搭建项目的过程中,遇到了很多问题,比如如何将接口请求代理到node层,如何同时启动前后端项目,如何实现模块热更新,如何美化界面,如何使图片加载更快等等。2小时内能够独立解决的问题,则独立解决,否则应该积极求助求助他人。同时,在求助他人的时候,应当准确,清晰的描述和定位问题,也便于给予帮助的人更快的理解,提高效率。

总之,毕业设计这个项目带给了我更多的思考,技术在不断发展,新旧之间有一个平衡,作为开发人员,如何站在用户的角度思考,为了实现某个功能,而采取最合适的技术,而非使功能适配某项技术。需要我不断的去探索和学习。同时,也十分感谢在完成毕业设计过程中,指导我的老师、帮助我的同学和前辈,没有你们的帮助,我的思考方向可能会截然不同,开发效率和项目质量可能会大打折扣。在此再次表示我诚挚的谢意!
